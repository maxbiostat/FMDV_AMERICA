\documentclass[a4paper,10pt]{article}
\usepackage[utf8]{inputenc}
\usepackage{graphicx}

\bibliographystyle{plos2009}%
%opening
\title{Assessing the impact of taxon sampling on phylogeographic reconstruction} % aargh, lousy name
\author{
Luiz Max de Carvalho, Nuno Faria, Guy Baele,\\ 
Andres Perez, Philippe Lemey and Waldemir Silveira 
}

\begin{document}

\maketitle

\section{Background}
The data sets used in this study presented high preferentiality of sampling, with some countries being overrepresented.
% A possibility, unexplored in this paper, is to sample each location with probability proportional to its disease prevalence
% In theory, a Bayesian approach should offer a certain degree of protection against model mispecification...
\section{Comparing rate matrices}
Sometimes \\
L1 and L2 norms
We thank Professor Marc A. Suchard (UCLA) for advice on this topic.\\
\section{Quantifying spatial signal extraction}
KL root for each subsample
\section{Bayesian Stochastic Search Variable Selection}
\section{Parameter estimation without the overrepresented locations}
\section{Parameter estimation with downsampling}
% For our experiments, we obtained five downsampled data sets for each serotype
\section{``Representation-informed'' priors}
\section{Simulating from the null}
\newpage
\bibliography{Text_S2}
\end{document}
